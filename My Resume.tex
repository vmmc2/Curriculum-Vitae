%%%%%%%%%%%%%%%%%%%%%%%%%%%%%%%%%%%%%%%
% Deedy - One Page Two Column Resume
% LaTeX Template
% Version 1.1 (30/4/2014)
%
% Original author:
% Debarghya Das (http://debarghyadas.com)
%
% Original repository:
% https://github.com/deedydas/Deedy-Resume
%
% IMPORTANT: THIS TEMPLATE NEEDS TO BE COMPILED WITH XeLaTeX
%
% This template uses several fonts not included with Windows/Linux by
% default. If you get compilation errors saying a font is missing, find the line
% on which the font is used and either change it to a font included with your
% operating system or comment the line out to use the default font.
% 
%%%%%%%%%%%%%%%%%%%%%%%%%%%%%%%%%%%%%%
% 
% TODO:
% 1. Integrate biber/bibtex for article citation under publications.
% 2. Figure out a smoother way for the document to flow onto the next page.
% 3. Add styling information for a "Projects/Hacks" section.
% 4. Add location/address information
% 5. Merge OpenFont and MacFonts as a single sty with options.
% 
%%%%%%%%%%%%%%%%%%%%%%%%%%%%%%%%%%%%%%
%
% CHANGELOG:
% v1.1:
% 1. Fixed several compilation bugs with \renewcommand
% 2. Got Open-source fonts (Windows/Linux support)
% 3. Added Last Updated
% 4. Move Title styling into .sty
% 5. Commented .sty file.
%
%%%%%%%%%%%%%%%%%%%%%%%%%%%%%%%%%%%%%%%
%
% Known Issues:
% 1. Overflows onto second page if any column's contents are more than the
% vertical limit
% 2. Hacky space on the first bullet point on the second column.
%
%%%%%%%%%%%%%%%%%%%%%%%%%%%%%%%%%%%%%%

\documentclass[]{deedy-resume-openfont}


\begin{document}

%%%%%%%%%%%%%%%%%%%%%%%%%%%%%%%%%%%%%%
%
%     LAST UPDATED DATE
%
%%%%%%%%%%%%%%%%%%%%%%%%%%%%%%%%%%%%%%

%%%%%%%%%%%%%%%%%%%%%%%%%%%%%%%%%%%%%%
%
%     TITLE NAME
%
%%%%%%%%%%%%%%%%%%%%%%%%%%%%%%%%%%%%%%


\namesection{Victor Miguel de Morais}{ Costa}
{\urlstyle{same}\href{mailto:vmmc2@cin.ufpe.br}{vmmc2@cin.ufpe.br} | +55 81 995272379}

%%%%%%%%%%%%%%%%%%%%%%%%%%%%%%%%%%%%%%
%
%     COLUMN ONE
%
%%%%%%%%%%%%%%%%%%%%%%%%%%%%%%%%%%%%%%

\begin{minipage}[t]{0.33\textwidth} 

%%%%%%%%%%%%%%%%%%%%%%%%%%%%%%%%%%%%%%
%     EDUCATION
%%%%%%%%%%%%%%%%%%%%%%%%%%%%%%%%%%%%%%

\section{Education} 

\subsection{UFPE - Federal University \newline of Pernambuco}
\descript{B.Sc. in Computer Engineering}
\location{Feb. 2018 - Dec. 2022 (expected)}
\location{Cum. GPA: 8.94/10.00}
\vspace{1mm} %5mm vertical space

\descript{Relevant Coursework}
\textbullet{}\location{Algorithms and Data Structures}
\textbullet{}\location{Computer Architecture}
\textbullet{}\location{Operating Systems}
\textbullet{}\location{Software-Hardware Interface}
%\textbullet{}\location{Data Science}
\vspace{2mm} %5mm vertical space

% \vspace{3mm} %5mm vertical space

%%%%%%%%%%%%%%%%%%%%%%%%%%%%%%%%%%%%%%
%     LINKS
%%%%%%%%%%%%%%%%%%%%%%%%%%%%%%%%%%%%%%

%\section{Links} 
%Github:// \href{https://github.com/deedydas}{\custombold{deedydas}} \\
%LinkedIn://  \href{https://www.linkedin.com/in/ladsongomes}{\custombold{ladsongomes}} \\
%\sectionsep

%%%%%%%%%%%%%%%%%%%%%%%%%%%%%%%%%%%%%%
%     COURSEWORK
%%%%%%%%%%%%%%%%%%%%%%%%%%%%%%%%%%%%%%

%\section{Coursework}
%\subsection{Graduate}
%Advanced Machine Learning \\
%Open Source Software Engineering \\
%Advanced Interactive Graphics \\
%Compilers + Practicum \\
%Cloud Computing \\
%\sectionsep

%\subsection{Undergraduate}
%Information Retrieval \\
%Operating Systems \\
%Artificial Intelligence + Practicum \\
%Functional Programming \\
%Computer Graphics + Practicum \\
%{\footnotesize \textit{\textbf{(Research Asst. \& Teaching Asst) }}} \\
%Unix Tools and Scripting \\
%\sectionsep

\section{Links}
%\location{Personal page: \href{https://thiagoaugustosm.com}{thiagoaugustosm.com}}
\location{Github: \href{https://github.com/vmmc2}{/vmmc2}}
%\location{LinkedIn: \href{https://www.linkedin.com/in/hugo97s/}{/hugo97s}}
% \sectionsep

%%%%%%%%%%%%%%%%%%%%%%%%%%%%%%%%%%%%%%
%     SKILLS
%%%%%%%%%%%%%%%%%%%%%%%%%%%%%%%%%%%%%%

\section{Skills}
\vspace{2mm} %5mm vertical space

%\subsection{Core}
%Public Speaking \textbullet{} Team Management\\
%IoT \textbullet{} Mobile and Web devlopment 
%\vspace{3mm} %5mm vertical space

\subsection{Programming}
\location{Fluent:}
 C \textbullet{} C++ \textbullet{} Dart\textbullet{} Python \newline
\location{Familiar:}
\vspace{1mm}
Kotlin \textbullet{} SystemVerilog \newline Assembly x86 \textbullet{} Verilog
\vspace{3mm} %5mm vertical space

%\subsection{Embedded Prototyping \\ \& Tools}
%Arduino \textbullet{} ESP8266 \\
%ESP32 \textbullet{} LoRa
%\vspace{3mm} %5mm vertical space

\subsection{Frameworks}
Bulma \textbullet{} Django
\vspace{3mm}

\subsection{Technologies}
Git
\vspace{3mm}

\subsection{Languages}
\location{Portuguese (Native)}
\location{English (Advanced)}
\vspace{3mm}

\subsection{Awards}
\textbullet \location{4th Place - Microsoft Hackathon "Bot-A-ndo" a mão na massa}

\sectionsep

% \section{WorkShops}
% \vspace{2mm} %5mm vertical space
% \descript{How to create an amazing experience using P5.js}
% \location{JSDay Recife}
% \location{Dec. 2018 | Recife - PE, BR}
% \textbullet{} Presented about P5.js' features and advantages in developing graphic, audio and text interaction on browser.


%%%%%%%%%%%%%%%%%%%%%%%%%%%%%%%%%%%%%%
%
%     COLUMN TWO
%
%%%%%%%%%%%%%%%%%%%%%%%%%%%%%%%%%%%%%%

\end{minipage} 
\hfill
\begin{minipage}[t]{0.66\textwidth} 

%%%%%%%%%%%%%%%%%%%%%%%%%%%%%%%%%%%%%%
%     EXPERIENCE
%%%%%%%%%%%%%%%%%%%%%%%%%%%%%%%%%%%%%%

\section{Experience}

\runsubsection{RobôCIn}
\descript{| Undergraduate Researcher }
\location{Nov. 2019 – Present | Recife - PE, BR}
\vspace{\topsep} % Hacky fix for awkward extra vertical space
\begin{tightemize}
\item RobôCIn is a research group of the Informatics center at UFPE which resolve problems using
robotics and innovative solutions using A.I, computer vision, mechanics and electronics.
\item Currently working in the 2D-Simulation division and focused on developing the strategy of the agents of our team for the next competitions.
\end{tightemize}


\runsubsection{PETLAB}
\descript{| Undergraduate Researcher }
\location{Aug. 2019 – Nov. 2019 | Recife - PE, BR}
\begin{tightemize}
\item PETLAB is a program developed in a partnership between PET(Programa de Educação Tutorial) and
 three research and development laboratories present at UFPE: Voxar Labs, LIKA and SPG. I worked as a developer and learned about computer vision, OpenCV and  different applications of cellular automata.
\item Relevant technologies used: Python, C++.
\end{tightemize}
% \vspace{\topsep} % Hacky fix for awkward extra vertical space

\runsubsection{UFPE}
\descript{| Vector and Linear Algebra Assistant Teacher }
\location{Aug. 2018 – Aug. 2019 | Recife - PE, BR}
\begin{tightemize}
\item Taught tutoring sessions for the students. Also made presentations that were used by the students as a way for them to prepare for the exams.
\item Prepared a series of mini tests that were used as a part of the final grade of the students.
\end{tightemize}

% \runsubsection{SENAI - Federal Institute of Education, Sciences and Technology of Pernambuco}
% \descript{| Technician Researcher }
% \location{Feb. 2016 – Dec. 2016 | Recife - PE, BR}
% \begin{tightemize}
% \item Worked at the RailBee Research Group developing the RailBee System, a Telemetric System for Strategic Monitoring of Urban Trains.
% \item Developed SOFIA, a system for managing, processing and analyzing real-time data of RailBee System.
% \item Managed the technology transfer of RailBee System from IFPE to CBTU (Brazilian Company of Urban Trains).
% \end{tightemize}




%\sectionsep
% \runsubsection{\href{http://ifs.edu.br/}{Instituto Federal de Sergipe - IFS}}
% \descript{| Network Admin Intern}
% \location{Jan. 2013 - Feb. 2015 | Aracaju - SE, BR}
% %\vspace{\topsep} % Hacky fix for awkward extra vertical space
% \begin{tightemize}
% \item Had hands-on learning with the maintenance and management of Linux, BSD and Windows systems, both on bare metal and virtualized through ESXi and Hyper-V.
% \item Automated tasks with Python and shell scripts in a production environment. 
% \item Kept track of the internal network's performance and stability through Zabbix.
% \item Managed to detect network downtimes faster while also decreasing their occurrences, increasing overall campus productivity.
% \end{tightemize}

%\sectionsep

\section{Projects}

\runsubsection{CInGames}
\descript{| A project focused on integrating games and FPGA }
\location{Oct. 2019 - Dec. 2019| Recife - PE, BR}
\begin{tightemize}
\item {CInGames is a project that contains 3 different games(Pong, Minesweeper, Genius) and these games can be played at the Altera DE2i-150 FPGA Board. Worked on the development of the Genius game and was also responsible to develop the Driver, so that these games could be played at the board. }
\item {Relevant technologies used: C/C++, MakeFile.}
\end{tightemize}

\runsubsection{Gasolina: Greve Infinita}
\descript{| A multiplayer battle-royale game }
\location{May. 2018 - Jul.2018 | Recife - PE, BR}
\begin{tightemize}
\item {Gasolina: Greve Infinita is a Battle-Royale game that can be played by at most 4 players in a local server. After the development, the game was tested by freshman students.}
\item {Responsible for the game art and for the development of the connection server-client. Relevant technologies used: C.}
\end{tightemize}

\runsubsection{Competitive Programming}
\descript{| A repository for Competitive Programming}
\location{Aug. 2018 - Present | Recife - PE, BR}
\begin{tightemize}
\item {Competitive Programming is a repository that I made which contains solutions for several problems about algorithms and data structures from different online judges such as: Codeforces, UVa, SPOJ and more. It was made for educational purpose and to help other students in order to solve a specific problem.}
\item {Relevant technologies used: C, C++, Python.}
\end{tightemize}

\runsubsection{Nemesis}
\descript{| A cache simulator}
\location{Jul. 2019 - Aug. 2019 | Recife - PE, BR}
\begin{tightemize}
\item {Nemesis is a cache simulator of 1024 words made using Python. It can simulate 16 different types of cache and supports both write access and read access. It was mainly developed using the RISC-V architecture as a basis and inspiration.}
\item {Relevant technologies used: Python.}
\end{tightemize}

%\sectionsep

%\section{Workshops}
%\vspace{0.5mm} %5mm vertical space
%\descript{Chatbots | Building your first chatbot with dialogflow}
%\location{Campus Party Natal}
%\location{April 2018 | Natal - RN, BR}
%\textbullet{} Presentation of theoretical contents about Chatbots, tips of good %practices when starting to plan
%a ChatBot, Overview of the Google ChatBots platform: DialogFlow; and prototyping the ChatBot by the participants.

%\vspace{0.5mm} %5mm vertical space
%\descript{IoT introduction with NodeMcu}
%\location{Campus Party Bahia}
%\location{May 2018 | Salvador - BA, BR}
%\textbullet{} Presented and introduced the participants to development with NodeMcu architecture using ESP8266, a compact and versatile IoT platform.
% \descript{Chatbots | Building your first chatbot with dialogflow}
% \location{Campus Party Natal | Campus Party Bahia}
% \location{Dec. 2018 | Recife - PE, BR}
% \textbullet{} Presented about P5.js' features and advantages in developing graphic, audio and text interaction on browser.


%\sectionsep

% \runsubsection{Ocular sensor for computers}
% \descript{| Co-Founder | Developer }

% \location{Jun. 2014 - Jul. 2015 | Aracaju - SE, BR}
% \begin{tightemize}
% \item {Creation of a low cost optical sensor that could capture and translate the movements made by the pupils into mouse movements.}
% \item {Developed both the firmware and desktop software using C++ and OpenCV.}
% \item {Eye tracking hardware made with an Arduino Uno attached to a PSEye camera.}
% \item {With it, people with disabilities regarding the superior limbs were able to use a computer without the need for an expensive device.}
% \end{tightemize}

%\sectionsep



%%%%%%%%%%%%%%%%%%%%%%%%%%%%%%%%%%%%%%
%     SOCIETIES
%%%%%%%%%%%%%%%%%%%%%%%%%%%%%%%%%%%%%%

%\section{Societies} 

%\begin{tabular}{rll}
%2014 	& top 12\%ile    & Tau Beta Pi Engineering Honor Society\\
%2014   & National   & The Global Leadership and Education Forum (tGELF)\\
%2012   &  National  & Golden Key International Honor Society\\
%2012   &  National   & National Society of Collegiate Scholars\\
%\end{tabular}
%\sectionsep

\end{minipage} 
\end{document}
